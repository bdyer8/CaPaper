Solid-state Resetting
The clumped isotope method also allows investigation of the phenomenon of isotopic resetting
in carbonates at temperatures higher or lower than those of formation or diagenetic alteration.
This article is protected by copyright. All rights reserved.
For example, does a calcite or dolomite mineral formed at sedimentary temperatures, change
its ratio of 13C?18O bonds at higher temperature and, if so, at what temperature and at what
rate? Alternatively, do minerals formed at high temperatures continually reorganize their 13C?
18O bonds until a ratio is locked in at lower temperatures? Such a phenomenon would be
important because it would constrain the burial history of a carbonate and perhaps the original
depositional temperatures could be back calculated assuming that there was information on
the rate of change at specific temperatures and the burial temperature. If solid diffusion was
responsible for these changes, then the ?18Ow calculated using such temperatures would in fact
be erroneous unless the temperatures were corrected for the burial history. Studies
investigating this phenomenon indicate that such resetting does in fact take place at
geologically significant rates, particularly at temperatures higher than 300oC (Henkes et al.,
2014; Passey & Henkes, 2012).

Diagenetic processes have been
noted by some workers (Steuber & Buhl, 2006), the most important of which is fractionation
during precipitation of biogenic carbonates, although there is minimal fractionation of 44Ca
during dissolution and precipitation reactions (Fantle & DePaolo, 2007). This allows the ?44Ca
value to be used to calculate rates of recrystallization.

For example,
cements often form between grains without affecting the chemical composition of the grains
themselves, or different components may dissolve and reprecipitate preferentially. In addition
crystals forming from the same solution may exhibit different trace element and isotopic
compositions in different growth sectors (Dickson, 1991; Reeder & Grams, 1987; Reeder &
Paquette, 1989; TenHave & Heijnen, 1985) (compositional sector zoning). Bulk stable isotopic
measurements, which are common in many studies, may not capture the chemical signatures
of diagenetic processes unless the entire rock has been altered. The microsampling of
carbonates has progressed from the initial attempts of Dickson & Coleman (1980) who used a
scalpel to excise sufficient material from thin sections, to more modern use of computercontrolled
microdrilling methods. However, these methods are still rather crude and do not
offer the precision necessary to sample material at spatial resolutions of less than ca 100 to 200
?m. 


As
stated by Hudson (1977): ??we must take our limestone to pieces? if we want to adequately
understand the diagenesis of the rock.


Rainwater contains only small quantities of dissolved salts, derived from aerosols and
atmospheric CO2. The high CO2 and low concentration of Ca2+ causes this water to be corrosive
to all carbonate minerals. Once in the vadose zone, the rainwater (now groundwater) acquires
additional CO2 from the decay and respiration of local OM and initially dissolves the local
calcium carbonate until saturation is attained with respect to the ambient carbonate
mineralogy. In geologically young terrains, the carbonate sediments are mainly composed of

Because these minerals are less stable than LMC, precipitation of LMC can
occur before dissolution of aragonite and HMC is complete (Budd, 1988). Precipitation of LMC
causes further undersaturation with respect to aragonite and HMC and the system could
theoretically dissolve the metastable minerals and precipitate the more stable ones until all the
aragonite and HMC are consumed. Further additions of freshwater to the system will then
result in dissolution of LMC, at least in the upper portion of the aquifer.

The recognition that diagenetic
carbonates acquire the ?18O signature of their recrystallizing fluids is the basis for numerous
other interpretations of the behaviour of ?18O values in diagenetic carbonates.

gross and tracey meteoric diagenesis soil organic matter

sandberg1975-recrystallization

kim and oneill - burial diagenesis